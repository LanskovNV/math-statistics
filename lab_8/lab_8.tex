%%%
% set up document type
%%%
\documentclass[12pt]{article}

%%%
% declare all packages
%%%
\usepackage[left=25mm, top=20mm, right=25mm, bottom=30mm,nohead,nofoot]{geometry} 

\usepackage[T2A]{fontenc}
\usepackage[utf8]{inputenc}
\usepackage[english, russian]{babel}

\usepackage{graphics, graphicx}

\usepackage{url}
\usepackage{hyperref}

\usepackage{amssymb,latexsym} 
\usepackage{MnSymbol}
\usepackage{mathrsfs}

\usepackage[nottoc,numbib]{tocbibind}
\usepackage{float}
\usepackage{listings}
\usepackage{multirow}
\usepackage{hhline}

\usepackage{color,colortbl}

%%%
% document settings
%%%
\setcounter{tocdepth}{4}
\graphicspath{ {./pic/} }

\renewcommand{\listoffigures}{\begingroup  % add number to list of graphics
\tocsection
\tocfile{\listfigurename}{lof}
\endgroup}
\renewcommand{\listoftables}{\begingroup  % add number to list of tables
\tocsection
\tocfile{\listtablename}{lot}
\endgroup}

%******************************************************************
%******************************************************************
\begin{document}

\begin{titlepage}
	\center
		Санкт-Петербургский Политехнический 
		университет Петра Великого
		Институт прикладной математики и механики
		\\ \textbf{Кафедра «Прикладная математика»}

	\vfill ~
	\textbf{
		\\ \large ЛАБОРАТОРНАЯ РАБОТА №8
	}
	\\	по дисциплине 
	\\	"Математическая статистика"

	\vfill ~

	Выполнил студент гр. \textbf{33631/1} \\
	\textbf{Лансков.Н.В.} \\ 

\vfill

{\large}	Санкт-Петербург
\\ 2019
\end{titlepage}

%%%
% Table of conetnts 
%%%

\tableofcontents 
\newpage
% \listoffigures
% \newpage
\listoftables
\newpage

%%%
% Text
%%%
\section{Постановка задачи}

Для двух выборок $20$ и $100$ элементов, сгенерированных согласно нормальному закону $N(x,0,1),$ для параметров масштаба и положения построить асимптотически нормальные интервальные оценки на основе точечных оценок метода максимального правдоподобия и классические интервальные оценки на основе статистик $\chi^2$ и Стьюдента. В качестве параметра надёжности взять $\gamma = 0.95.$


\section{Теория}

Оценкой максимального правдоподобия для математического ожидания  является среднее арифметическое: $\mu=\frac{1}{n}\sum\limits_{i=1}^nx_i.$

Оценка максимального правдоподобия для дисперсии вычисляется по формуле: $\sigma^2 = \frac{1}{n}\sum\limits_{i=1}^n(x_i-\overline{x})^2.$

Доверительным интервалом или интервальной оценкой числовой характеристики или параметра распределения $\theta$ с доверительной вероятностью $\gamma$ называется интервал со случайными границами $(\theta_1,\theta_2),$ содержащий параметр $\theta$ с вероятностью $\gamma$ \cite{8_1}.

Функция распределения Стьюдента \cite{8_2}:
\begin{equation}
    T = \sqrt{n-1}\frac{\overline{x}-\mu}{\delta}
\end{equation}

Функция плотности распределения $\chi^2$ \cite{8_3}:
\begin{equation}
    f(x) = \begin{cases}
    0,&x\leq 0\\
    \frac{1}{2^\frac{n}{2}\Gamma\left(\frac{n}{2}\right)}x^{\frac{n}{2}-1}e^{-\frac{x}{2}},& x>0
    \end{cases}
\end{equation}

Интервальная оценка математического ожидания \cite{8_4}:
\begin{equation}
    P=\left(\overline{x}-\frac{\sigma t_{1-\frac{a}{2}}(n-1)}{\sqrt{n-1}}<\mu<\overline{x}+\frac{\sigma t_{1-\frac{a}{2}}(n-1)}{\sqrt{n-1}}\right) = \gamma,
\end{equation}
где $t_{1-\frac{a}{2}}\;\--$ квантиль распределения Стьюдента порядка $1-\frac{a}{2}.$

Интервальная оценка дисперсии \cite{8_2}:
\begin{equation}
    P=\left(\frac{\sigma\sqrt{n}}{\sqrt{\chi^2_{1-\frac{a}{2}}(n-1)}}<\sigma<\frac{\sigma\sqrt{n}}{\sqrt{\chi^2_\frac{a}{2}(n-1)}}\right) = \gamma,
\end{equation}
где $\chi_{1-\frac{a}{2}}^2,\;\chi_\frac{a}{2}^2\;\--$ квантили распределения Стьюдента порядков $1-\frac{a}{2}$ и $\frac{a}{2}$ соответственно.

Асимптотическая интервальная оценка математического ожидания \cite{8_2}:
\begin{equation}
    P = \left(\overline{x}-\frac{\sigma u_{1-\frac{a}{2}}}{\sqrt{n}}<\mu<\overline{x}+\frac{\sigma u_{1-\frac{a}{2}}}{\sqrt{n}}\right)=\gamma,
\end{equation}
где $u_{1-\frac{a}{2}}\;\--$ квантиль нормального распределения $N(x,0,1)$ порядка $1-\frac{a}{2}.$


\section{Реализация}
Работы была выполнена на языке $Python 3.7.$
Для генерации выборок использовался модуль \cite{numpy}.
Для построения графиков использовалась библиотека matplotlib \cite{plotlib}.
Функции распределения обрабатывались при помощи библиотеки scipy.stats \cite{skp}


\section{Результаты}

\begin{table}[H]
\caption{Результаты}
\label{tab:my_label1}
\begin{center}
\vspace{5mm}
\begin{tabular}{|c|c|c|c|}
\hline
Метод & $n$&$\mu$&$\sigma$\\
\hline
&$20$&	$[-0.64221, 0.33978]$&		$[0.94061, 1.80746]$ \\
\cline{2-4}
\raisebox{1.5ex}[0cm][0cm]{На основе ММП}&100&	$[-0.21225, 0.25872]$&	$[1.036808, 1.37085]$\\
\hline
&20&	$[-0.69368, 0.39125]$&		$[0.98131, 1.49417]$ \\
\cline{2-4}
\raisebox{1.5ex}[0cm][0cm]{Асимптотический}&100	&$[-0.20821, 0.25468]$&	$[1.02183, 1.33989]$\\
\hline
\end{tabular}
\end{center}
\end{table}


\section{Выводы}

По полученным результатам видно, что оба подхода дают лучший результат на выборках большого объема. Если рассматривать результаты для выборки объема $n=20$ элементов, то видно, что интервал меньше и точнее в классической интервальной оценке.

\section{Приложения}

Исходники: \url{https://github.com/LanskovNV/math_statistics/tree/master/lab_8}

%%%
% Literature
%%%
\begin{thebibliography}{}
    \bibitem{numpy}  Модуль numpy  -  https://physics.susu.ru/vorontsov/language/numpy.html
    
    \bibitem{plotlib} 
    Модуль matplotlib - https://matplotlib.org/users/index.html
    
    \bibitem{skp}
    Модуль scipy - https://docs.scipy.org/doc/scipy/reference/
    

\bibitem{8_1}
https://en.wikipedia.org/wiki/Confidence\_interval

\bibitem{8_2}
https://en.wikipedia.org/wiki/Student\%27s\_t-distribution

\bibitem{8_3}
https://en.wikipedia.org/wiki/Chi-squared\_distribution

\bibitem{8_4}
Шевляков Г. Л. Лекции по математической статистике, 2019.

\end{thebibliography}

\end{document}

