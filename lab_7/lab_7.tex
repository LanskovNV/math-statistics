\documentclass[12pt]{article}
\usepackage[T2A]{fontenc}
\usepackage[utf8]{inputenc}
\usepackage{graphics}
\usepackage[english, russian]{babel}
\usepackage{csquotes}
\usepackage{graphicx}
\usepackage{amsmath}
\usepackage{longtable}
\usepackage[left=25mm, top=20mm, right=25mm, bottom=30mm,nohead,nofoot]{geometry}
\usepackage{verbatim}
\usepackage{hyperref}
\usepackage{amssymb,latexsym}  % Standard packages
\usepackage{MnSymbol}
\usepackage{mathrsfs}
\usepackage{amsthm}
\usepackage{indentfirst}
\usepackage[nottoc,numbib]{tocbibind}
\usepackage{float}

%******************************************************************
%******************************************************************

\setcounter{tocdepth}{4}
\graphicspath{ {./pic/} }

\begin{document}

\begin{titlepage}
	\center
		Санкт-Петербургский Политехнический 
		университет Петра Великого
		Институт прикладной математики и механики
		\\ \textbf{Кафедра «Прикладная математика»}

	\vfill ~
	\textbf{
		\\ \large ЛАБОРАТОРНАЯ РАБОТА №7
	}
	\\	по дисциплине 
	\\	"Математическая статистика"

	\vfill ~

	Выполнил студент гр. \textbf{33631/1} \\
	\textbf{Лансков.Н.В.} \\ 

\vfill

{\large}	Санкт-Петербург
\\ 2019
\end{titlepage}

%%%
% Table of conetnts 
%%%
% \settocdepth{chapter}
\tableofcontents
\newpage
% \listoffigures
% \newpage
\listoftables
\newpage
\pagebreak

% \setcounter{chapter}{1}

%%%
% Text
%%%
\section{Постановка задачи}

Необходимо сгенерировать выборку объемом $100$ элементов для нормального распределения $N(x;0,1).$ По сгенерированной выборке оценить параметры $\mu$ и $\sigma$ нормального закона методом максимального правдоподобия. В качестве основной гипотезы $H_0$ будем считать, что сгенерированное распределение имеет вид $N(x,\overset{\wedge}{\mu},\overset{\wedge}{\sigma} ).$ Проверить основную гипотезу, используя критерий согласия $\chi$. В качестве ровня значимости взять $\alpha=0,05.$ Привести таблицу вычислений $\chi^2.$


\section{Теория}
\subsection{Метод максимального правдоподобия}
Метод максимального правдоподобия $\--$ метод оценивания неизвестного параметра путём максимимзации функции правдоподобия.
\begin{equation}
    \overset{\wedge}{\theta}_{\text{МП}}=argmax \mathbf{L}(x_1,x_2,\ldots,x_n,\theta)
\end{equation}

Где $\mathbf{L}$ это функция правдоподобия, которая представляет собой совместную плотность вероятности независимых случайных величин $X_1,x_2,\ldots,x_n$ и является функцией неизвестного параметра $\theta$
\begin{equation}
    \mathbf{L} = f(x_1,\theta)\cdot f(x_2,\theta)\cdot\cdots\cdot f(x_n,\theta)
\end{equation}
Оценкой максимального правдоподобия будем называть такое значение $\overset{\wedge}{\theta}_{\text{МП}}$ из множества допустимых значений параметра $\theta,$ для которого функция правдоподобия принимает максимальное значение при заданных $x_1,x_2,\ldots,x_n.$

Тогда при оценивании математического ожидания $m$ и дисперсии $\sigma^2$ нормального распределения $N(m,\sigma)$ получим:
\begin{equation}
    \ln(\mathbf{L})=-\frac{n}{2}\ln(2\pi)-\frac{n}{2}\ln\left(\sigma^2\right)-\frac{1}{2\sigma^2}\sum\limits_{i=1}^n(x_i-m)^2
\end{equation}

\subsection{Критерий согласия Пирсона}
Разобьём генеральную совокупность на $k$ неперсекающихся подмножеств $\Delta_1, \Delta_2,\ldots, \Delta_k,\;\Delta_i = (a_i,a_{i+1}],$ $p_i = P(X\in\Delta_i),\;i=1,2,\ldots,k\; \--$ вероятность того, что точка попала в $i$ый промежуток.

Так как генеральная совокупность это $\mathbb{R},$ то крайние промежутки будут бесконечными: $\Delta_1=(-\infty,a_1],\;\Delta_k=(a_k,\infty),\;p_i = F(a_i)-F(a_{i-1})$

$n_i\;\--$ частота попадания выборочных элементов в $\Delta_i,\;i=1,2,\ldots,k.$

В случае справедливости гипотезы $H_0$ относительно частоты $\frac{n_i}{n}$ при больших $n$ должны быть близки к $p_i,$ значит в качестве меры имеет смысл взять: 
\begin{equation}
    Z = \sum\limits_{i=1}^k\frac{n}{p_i}\left(\frac{n_i}{n}-p_i\right)^2
\end{equation}
Тогда
\begin{equation}
    \chi^2_B=\sum\limits_{i=1}^k\frac{n}{p_i}\left(\frac{n_i}{n}-p_i\right)^2=\sum\limits_{i=1}^k\frac{(n_i-np_i)^2}{np_i}
\end{equation}
Для выполнения гипотезы $H_0$ должны выполняться следующие условия \cite{7_2}:
\begin{equation}
    \chi_B^2 < \chi_{1-\alpha}^2(k-1)
\end{equation}
где $\chi_{1-\alpha}^2(k-1)\;\--$ квантиль распределения $\chi^2$ с $k-1$ степенями свободы порядка $1-\alpha,$ где $\alpha$ заданный уровень значимости.

\section{Реализация}
Работы была выполнена на языке $Python 3.7.$
Для генерации выборок использовался модуль \cite{numpy}.
Для построения графиков использовалась библиотека matplotlib \cite{plotlib}.
Функции распределения обрабатывались при помощи библиотеки scipy.stats \cite{skp}


\section{Результаты}
\subsection{Метод максимального правдоподобия}

При подсчете оценок параметров закона нормального распределения методом максимального правдоподобия были получены следующие значения:
\begin{equation}
\begin{split}
    &\overset{\wedge}{m}_{\text{МП}} = -0.1167\\
   &  \overset{\wedge}{\sigma}^2_{\text{МП}} = 0.9859
\end{split}
\end{equation}
\subsection{Критерий Пирсона}
\begin{table}[H]
\caption{Таблица вычислений $\chi^2$}
\label{tab:my_label1}
\begin{center}
\vspace{5mm}
\begin{tabular}{|c|c|c|c|c|c|c|}
\hline
 i & $\Delta_i$ & $n_i$ & $p_i$ & $np_i$ & $n_i-np_i$ & $\frac{(n_i-np_i)^2}{np_i}$\\
\hline
1&	 $(-\infty, -1.7100]$ &	4  &0.0530& 5.3025& -1.3025& 0.3199\\
\hline
2& (-1.7100, -0.8964)& 16& 0.1615& 16.1464& -0.1464& 0.0013\\
\hline
3& (-0.8964, -0.0828)& 35& 0.2992& 29.9200& 5.0800& 0.8625\\
\hline
4& (-0.0828, 0.7308)& 26& 0.2913& 29.1302& -3.1302& 0.3364\\
\hline
5& $(0.7308, \infty)$& 19& 0.1950& 19.5009& -0.5009& 0.0129\\
\hline
$\sum$&&		100&	1&	100&0&1.5330	\\

\hline
\end{tabular}
\end{center}
\end{table}

$$\chi_B^2= 1.5330$$




\section{Выводы}

В данной работе получено значение критерия согласия Пирсона $\chi_B^2 = 1.5330.$ Табличное значение квантиля  $\chi^2_{1-\alpha}(k-1)=\chi^2_{0.95}(4) = 9,4877$ \cite{chi_quant}.

Значит $\chi_B^2 < \chi^2_{0.95}(4),$ из этого следует, что основная гипотеза $H_0$ соотносится с выборкой на уровне $\alpha = 0.05.$

\begin{thebibliography}{}
    \bibitem{numpy}  Модуль numpy  -  https://physics.susu.ru/vorontsov/language/numpy.html
    
    \bibitem{plotlib} 
    Модуль matplotlib - https://matplotlib.org/users/index.html
    
    \bibitem{skp}
    Модуль scipy - https://docs.scipy.org/doc/scipy/reference/
    

\bibitem{7_2}
https://en.wikipedia.org/wiki/Pearson\%27s\_chi-squared\_test

\bibitem{chi_quant}
Таблица значений $\chi^2$ -  https://ru.wikipedia.org/wiki/\%D0\%9A\%D0\%B2\%D0\%B0\%D0\%BD\%D1\%82\%D0\%B8\%D0\%BB\%D0\%\do-B8\_\%D1\%80\%D0\%B0\%D1\%81\%D0\%BF\%D1\%80\%D0\%B5\%D0\%B4\%D0\%B5\%D0\%BB\%D0\%B5\%D0\%\do-BD\%D0\%B8\%D1\%8F\_\%D1\%85\%D0\%B8-\%D0\%BA\%D0\%B2\%D0\%B0\%D0\%B4\%D1\%80\%D0\%B0\%D1\%\do-82\%D0\%B2\%D0\%B0\%D0\%B4\%D1\%80\%D0\%B0\%D1\%82

\end{thebibliography}
\section{Приложения}

Исходники: \url{https://github.com/LanskovNV/math_statistics/tree/master/lab_7}

\end{document}

