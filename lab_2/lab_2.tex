%%%
% set up document type
%%%
\documentclass[12pt]{article}

%%%
% declare all packages
%%%
\usepackage[left=25mm, top=20mm, right=25mm, bottom=30mm,nohead,nofoot]{geometry} 

\usepackage[T2A]{fontenc}
\usepackage[utf8]{inputenc}
\usepackage[english, russian]{babel}

\usepackage{graphics, graphicx}

\usepackage{url}
\usepackage{hyperref}

\usepackage{amssymb,latexsym} 
\usepackage{MnSymbol}
\usepackage{mathrsfs}

\usepackage[nottoc,numbib]{tocbibind}
\usepackage{float}
\usepackage{listings}
\usepackage{multirow}
\usepackage{hhline}

\usepackage{color,colortbl}

%%%
% document settings
%%%
\setcounter{tocdepth}{4}
\graphicspath{ {./pic/} }

\renewcommand{\listoffigures}{\begingroup  % add number to list of graphics
\tocsection
\tocfile{\listfigurename}{lof}
\endgroup}
\renewcommand{\listoftables}{\begingroup  % add number to list of tables
\tocsection
\tocfile{\listtablename}{lot}
\endgroup}

%******************************************************************
%******************************************************************
\begin{document}

\begin{titlepage}
	\center
		Санкт-Петербургский Политехнический 
		университет Петра Великого
		Институт прикладной математики и механики
		\\ \textbf{Кафедра «Прикладная математика»}

	\vfill ~
	\textbf{
		\\ \large ЛАБОРАТОРНАЯ РАБОТА №2
	}
	\\	по дисциплине 
	\\	"Математическая статистика"

	\vfill ~

	Выполнил студент гр. \textbf{33631/1} \\
	\textbf{Лансков.Н.В.} \\ 

\vfill

{\large}	Санкт-Петербург
\\ 2019
\end{titlepage}

%%%
% Table of conetnts 
%%%

\tableofcontents 
\newpage
% \listoffigures
% \newpage
\listoftables
\newpage

%%%
% Text
%%%
\section{Постановка задачи}
Любыми средствами сгенерировать выборки размеров $20,$ $60,$ $100$ элементов для $5$ти распределений. Для каждой выборки вычислить $\overline{x},\; med\; x,\; Z_R,\; Z_Q,\; Z_{tr},$ при $r = \frac{n}{4}.$

Распределения:
\begin{enumerate}
\item Стандартное нормальное распределение
\item Стандартное распределение Коши
\item Распределение Лапласа с коэффициентом масштаба $\sqrt{2}$ и нулевым коэффициентом сдвига.
\item Равномерное распределение на отрезке $\left[-\sqrt{3}, \sqrt{3}\right]$
\item Распределение Пуассона со значением матожидания равным двум.
\end{enumerate}

\section{Теория}

\subsection{Выборочное среднее \cite{average}}

\begin{equation}
\overline{x} = \frac{1}{n}\sum_{i=1}^n x_i \hfill \label{eqn:average}
\end{equation}

\subsection{Выборочная медиана \cite{med}}

\begin{equation}
med\; x = \begin{cases}
x_{k+1}, & n = 2k+1\\
\frac{1}{2}\left(x_k+x_{k+1}\right), & n = 2k
\end{cases} \hfill  \label{eqn:med}
\end{equation}

\subsection{Полусумма экстремальных значений \cite{mean_extr}}

\begin{equation} 
Z_R = \frac{1}{2}\left(x_1+x_n\right) \hfill  \label{eqn:mean_extr}
\end{equation}

\subsection{Полусумма квартилей \cite{quartiles}}

\begin{equation}
Z_Q = \frac{1}{2}\left(Z_{\frac{1}{4}}+Z_{\frac{3}{4}}\right) \hfill  
\label{eqn:quartiles}
\end{equation}

\subsection{Усечённое среднее \cite{cut_mean}}

\begin{equation}
Z_{tr} = \frac{1}{n - 2r}\sum_{i=r+1}^{n-r} x_i \hfill  \label{eqn:cut_mean}
\end{equation}

\pagebreak

\section{Реализация}
Выполнено средствами \textit{python} c применением библиотеки \textit{numpy}\cite{numpy}

\section{Результаты}

\begin{table}[H]
\caption{normal}
\label{tab:my_label1}
\begin{center}
\vspace{5mm}
\begin{tabular}{|c|c|c|c|c|c|}
\hline
n = 20    &average     &med         &Zr          &Zq          &Ztr r = n/4 \\
\hline
E =       &0.003889    &-0.001204   &0.004247    &-0.002884   &0.014986    \\
\hline
D =       &0.050065    &0.069428    &0.135281    &0.056127    &0.060241    \\
\hline
n = 60    &average     &med         &Zr          &Zq          &Ztr r = n/4 \\
\hline
E =       &0.005639    &-0.001905   &0.006317    &0.003647    &-0.005183   \\
\hline
D =       &0.017071    &0.024751    &0.109666    &0.020065    &0.019645    \\
\hline
n = 100   &average     &med         &Zr          &Zq          &Ztr r = n/4 \\
\hline
E =       &-0.000990   &-0.008915   &-0.003211   &-0.007140   &0.000411    \\
\hline
D =       &0.010270    &0.015038    &0.091469    &0.012224    &0.011148    \\
\hline
\end{tabular}
\end{center}
\end{table}

\begin{table}[H]
\caption{cauchy}
\label{tab:my_label2}
\begin{center}
\vspace{5mm}
\begin{tabular}{|c|c|c|c|c|c|}
\hline
n = 20    &average     &med         &Zr          &Zq          &Ztr r = n/4 \\
\hline
E =       &0.574031    &-0.009207   &4.314604    &0.009490    &0.024569    \\
\hline
D =       &380.673006  &0.127344    &24864.038520&0.290641    &0.156087    \\
\hline
n = 60    &average     &med         &Zr          &Zq          &Ztr r = n/4 \\
\hline
E =       &16.889466   &0.007131    &-14.877379  &-0.006227   &-0.007330   \\
\hline
D =       &218415.033090&0.039049    &268917.253813&0.087454    &0.042852    \\
\hline
n = 100   &average     &med         &Zr          &Zq          &Ztr r = n/4 \\
\hline
E =       &-1.147046   &-0.000622   &-23.949387  &-0.002369   &-0.008024   \\
\hline
D =       &604.646590  &0.024850    &1995730.222185&0.051894    &0.023969    \\
\hline
\end{tabular}
\end{center}
\end{table}

\begin{table}[H]
\caption{laplace}
\label{tab:my_label3}
\begin{center}
\vspace{5mm}
\begin{tabular}{|c|c|c|c|c|c|}
\hline
n = 20    &average     &med         &Zr          &Zq          &Ztr r = n/4 \\
\hline
E =       &-0.001627   &-0.000952   &0.032669    &0.003535    &0.005730    \\
\hline
D =       &0.045803    &0.031761    &0.402496    &0.046401    &0.033325    \\
\hline
n = 60    &average     &med         &Zr          &Zq          &Ztr r = n/4 \\
\hline
E =       &-0.006856   &-0.002807   &0.050344    &-0.000216   &-0.000712   \\
\hline
D =       &0.016743    &0.009838    &0.431677    &0.017332    &0.009962    \\
\hline
n = 100   &average     &med         &Zr          &Zq          &Ztr r = n/4 \\
\hline
E =       &-0.000800   &0.001184    &-0.024119   &-0.002611   &-0.000909   \\
\hline
D =       &0.009861    &0.005529    &0.409545    &0.009740    &0.006101    \\
\hline
\end{tabular}
\end{center}
\end{table}

\begin{table}[H]
\caption{uniform}
\label{tab:my_label4}
\begin{center}
\vspace{5mm}
\begin{tabular}{|c|c|c|c|c|c|}
\hline
n = 20    &average     &med         &Zr          &Zq          &Ztr r = n/4 \\
\hline
E =       &-0.004788   &0.011925    &-0.000665   &-0.003460   &-0.002870   \\
\hline
D =       &0.049107    &0.134658    &0.013457    &0.071206    &0.097207    \\
\hline
n = 60    &average     &med         &Zr          &Zq          &Ztr r = n/4 \\
\hline
E =       &0.001583    &-0.004005   &-0.001657   &-0.005769   &-0.003667   \\
\hline
D =       &0.016670    &0.045087    &0.001706    &0.024632    &0.033966    \\
\hline
n = 100   &average     &med         &Zr          &Zq          &Ztr r = n/4 \\
\hline
E =       &0.000676    &0.004193    &-0.000025   &-0.005675   &0.004133    \\
\hline
D =       &0.010255    &0.028780    &0.000621    &0.015457    &0.019007    \\
\hline
\end{tabular}
\end{center}
\end{table}

\begin{table}[H]
\caption{poisson}
\label{tab:my_label5}
\begin{center}
\vspace{5mm}
\begin{tabular}{|c|c|c|c|c|c|}
\hline
n = 20    &average     &med         &Zr          &Zq          &Ztr r = n/4 \\
\hline
E =       &2.016900    &1.868000    &2.531500    &1.899250    &1.865800    \\
\hline
D =       &0.099684    &0.179576    &0.290758    &0.133599    &0.115410    \\
\hline
n = 60    &average     &med         &Zr          &Zq          &Ztr r = n/4 \\
\hline
E =       &2.005733    &1.932500    &2.945000    &1.936875    &1.843400    \\
\hline
D =       &0.033034    &0.054194    &0.233475    &0.034812    &0.041836    \\
\hline
n = 100   &average     &med         &Zr          &Zq          &Ztr r = n/4 \\
\hline
E =       &1.997070    &1.961500    &3.128000    &1.963625    &1.844780    \\
\hline
D =       &0.020288    &0.033268    &0.217616    &0.017067    &0.027989    \\
\hline
\end{tabular}
\end{center}
\end{table}

\section{Обсуждение}

\par При вычислении средних значений пришлось отбрасывать некоторое число знаков после запятой, так как дисперсия может гарантировать порядок точности среднего значения только до первого значащего знака после запятой в дисперсии включительно. Единственное исключение - стандартное распределение Коши, так как оно имеет бесконечную дисперсию, а значит не может гарантировать никакой точности.

\section{Выводы}

\par В процессе работы вычислены значения характеристик положения для определённых распределений на выборках фиксированной мощности и получено следующее ранжирование характеристик положения:

\begin{enumerate}
    \item Стандартное нормальное распределение $$\overline{x} < Z_{tr} < Z_Q < med\;x < Z_R$$
    
    \item Стандартное распределение Коши $$med\;x < Z_Q < Z_{tr} < \overline{x} < Z_R$$
    
    \item Распределение Лапласа (коэффициент масштаба $\sqrt{2}$ коэффициент сдвига равен нулю) $$med\;x < Z_{tr} < \overline{x} < Z_Q < Z_R$$
    
    \item Равномерное распределение на отрезке $\left[-\sqrt{3},\sqrt{3}\right]$ $$Z_R < \overline{x} < Z_{tr} < Z_Q < med\;x$$
    
    \item Распределение Пуассона (значение мат ожидания равно $3$) $$\overline{x} < Z_{tr} < Z_Q < med\;x < Z_R$$
    
\end{enumerate}

\section{Приложения}

Исходники: \url{https://github.com/LanskovNV/math_statistics/tree/master/lab_2}

\newpage

%%%
% Literature
%%%
\begin{thebibliography}{}
    \bibitem{numpy}  Модуль numpy  -  https://physics.susu.ru/vorontsov/language/numpy.html
    
    \bibitem{average}  
    Выборочное среднее  -  https://en.wikipedia.org/wiki/Sample\_mean\_and\_covariance
    
    \bibitem{med}  
    Выборочная медиана  -  http://femto.com.ua/articles/part\_1/2194.html
    
    \bibitem{mean_extr}  
    Полусумма экстремальных значений  -  https://studopedia.info/8-56888.html
    
    \bibitem{quartiles}  
    Квартили  -  https://studfiles.net/preview/2438125/page:13/
    
    \bibitem{cut_mean}  Усечённое среднее  -  https://ole-olesko.livejournal.com/15773.html
\end{thebibliography}

\end{document}

